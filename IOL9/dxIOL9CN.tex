\def \thisling{中文}
\def \thislang{中文}
\def \thistext{中文文本}
\def \olympiad{第九届国际语言学奥林匹克竞赛}
\def \xcountry{美国}
\def \yvillage{匹兹堡}
\def \Julyname{7月}
\def \olydates #1#2#3#4{#4年#3#1 — #2日}
\def \probindl{个人赛题目}
\def \solsindl{个人赛解答}
\def \probteam{团体赛题目}
\def \soluteam{团体赛解答}
\def \Teamword{团队}
\def \probword #1{题 \##1}
\def \pontword{分}
\def \regulats{解答规则}
\def \regulado{毋需抄题}
\def \regulare{将不同问题的解答分述于不同的答题纸上}
\def \regulami{每张纸上注明题号、座位号和姓名}
\def \towarrant{否则答题纸可能被误放或遗失}
\def \regulaty{解答需详细论证}
\def \regulatz{无解释之答案, 即便完全正确, 也会被处以低分}
\def \editorsz{编者}
\def \edinchef{主编}
\def \goodluck{祝你好运}
\def \rulesmot{规则}
\def \answersp{答案}
\def \andtrans{及其汉语翻译}
\def \chaotict{(乱序排列)}
\def \fordinsg #1{翻译成#1}
\def \corrcorr{请将其一一对应}
\def \filmties{请填补空缺}
\def \transall{若有多种可能的翻译, 请全部给出}
\def \isavowel{是元音}
\def \arvowels{\isavowel}
\def \atwovocs{\isavowel}
\def \isaconst{是辅音}
\def \aconsons{\isaconst}
\def \atwocons{\isaconst}
\def \aeligatu #1{#1 $\approx$~英语 \word{crack} 中的 \word{a}}
\def \eshiroko #1{#1 $\approx$~英语 \word{bed} 中的 \word{e}}
\def \cyqueska #1{#1~= \word{k}}
\def \jotsound #1{#1~= 普通话\word{叶}(\word{yè})中的\word{y}}
\def \wawsound #1{#1~= 普通话\word{文}(\word{wén})中的\word{w}}
\def \chaffric #1{#1 $\approx$~普通话\word{七} (\word{qī})中的\word{q}}
\def \glotstop #1{#1 是辅音(声门塞音)}
\def \tonmarks #1{标记 ~#1 表示声调}
\def \longmark #1{标记 ~#1~ 表示长元音}
\def \camacron{称为“长音符”}
\def \belongsto #1#2{#1隶属于#2}
\def \toCMande{曼德语族的中部语支}
\def \toNGerma{日尔曼语的北部亚语族}
\def \spokenca #1#2{#2, 约有\ #1人使用该语言}
\def \inLbrSle{在利比里亚和塞拉利昂}
\def \iFaroetc{在法罗群岛等地}
\def \thelgMez{梅诺米尼语}
\def \introMez{下列是梅诺米尼语的动词形式}
\def \mezvestr{梅诺米尼语动词的结构如下所示}
\def \rewheMez{梅诺米尼印第安人居于美国威斯康星州}
\def \retheMez #1{其人口数约为#1}
\def \elderMez{但仅有其中几十位年长者仍然使用这种以阿尔冈琴族名命名的语言}
\def \altheMez{尽管近来该语言的教学和使用有所扩展}
\def \ThelgVai{瓦伊语}
\def \thelgVai{瓦伊语}
\def \introVai{下列是瓦伊语短语}
\def \existerr #1{瓦伊语短语#1含有一处错误}
\def \corrtran #1{请将其更正, 并将该短语译成#1}
\def \erstsyll{在第一音节}
\def \bnvowels{在元音间}
\def \inN{(名词)}
\def \inV{(动词)}
\def \applerst{应用下列首条适合的规则}
\def \ThelgFar{法罗语}
\def \inFaroes{下列是用规则正字法书写的法罗语单词, 对应的语音表记}
\def \fillgaps{填补空缺}
\def \descruls{描述题中涉及的规则}
\def \indetscr{在语音表记中}
\def \theNahua{纳瓦特尔语}
\def \inNahua{下列为纳瓦特尔语单词}
\def \reNahua{古典纳瓦特尔语是墨西哥阿兹特克帝国的语言}
\def \kenaq{咱们 (我们和你们)}
\def \rootwAEp{开始}
\def \kewAEpeqtaq{开始}
\def \wAEpAhpew{开始笑}
\def \kewAEpAnaehkaeq{开始挖}
\def \wAEpohnaew{开始走}
\def \newAEpahtan{开始吃它}
\def \rootkaw{下}
\def \nekAwAhpem{笑翻}
\def \nekAweqtam{躺下}
\def \kekAwaeq #1{#1把它摊平}
\def \kawam #1{#1砍它}
\def \rootkEsk{破}
\def \kekEskaeq #1{#1把它劈破, 打破}
\def \kEskam #1{#1把它打破}
\def \kekEskahtaeq{把它咬破}
\def \rootket{出}
\def \nekAEtan #1{#1把它撬出}
\def \ketam #1{#1把它拿出}
\def \kekAEtohnaeq{走出}
\def \ketOhnaew{走出}
\def \rootpahk{掉}
\def \pahkAEsam{把它切掉}
\def \nepAhkaln #1{#1把它打掉, 撕掉}
\def \rootpAhk{(打) 开}
\def \pAhkeqtaw{打开}
\def \pAhkam #1{#1打开它}
\def \nepAhkarn #1{#1打开它}
\def \rootpIt{到这里}
\def \nepItohnaem{走过来 (到这里)}
\def \kepItahtaed{来吃它}
\def \kepItahtaeg{把它放进嘴里}
\def \pItam{把它递过来}
\def \roottaw{(刺) 穿}
\def \tawAnaehkaew{挖洞}
\def \tawAEsam{把它切一个洞}
\def \ketAwahtaeq{把它咬一个洞}
\def \netAwan #1{#1把它刺穿}
\def \rootwack{绕}
\def \newAckesan{把它绕着切}
\def \wackOhnaew{绕着走}
\def \mezen{用手}
\def \mezah{用工具}
\def \suffaht{用嘴}
\def \suffes{切}
\def \suffAhpe{笑}
\def \sAnaehkae{挖}
\def \suffohnae{走}
\def \pAhkaheg{开盖子或开门}
\def \Vintrans{不及物动词}
\def \Vtransit{及物动词}
\def \iambegin{若2个语素首元音皆为短元音, 第2个变为长元音}
\def \eyzur{财富}
\def \logi{火焰}
\def \lwgur{液体}
\def \skOgur{森林}
\def \swga{故事}
\def \toygur{吞咽}
\def \vizur{木材}
\def \bwga{雌禽}
\def \glaza{旋风}
\def \glwzur{余烬}
\def \hugi{心智}
\def \koyla{裂缝}
\def \lega{床}
\def \mOza{肉/鱼锅里的泡沫}
\def \plAga{困扰}
\def \skazi{损失}
\def \vAgur{海湾}
\def \tegi{保持沉默}
\def \trUgi{但愿(他)危害}
\def \mi{我}
\def \deyzi{杀}
\def \rAzi{建议}
\def \ta{他}
\def \knozar{揉捏}
\def \tregar{受伤}
\def \vegur{抬起}
\def \nad{他们}
\def \gleza{使高兴}
\def \kvwza{唱}
\def \mugu{必须}
\def \rwza{说}
\def \rUma{包含}
\def \spreiza{传播}
\def \viga{称重}
\def \wga{恐吓}
\def \me{我们}
\def \nyimii{蛇}
\def \ja{眼睛}
\def \kafa {肩膀}
\def \lEndE{容器}
\def \gbomus #1{鱼的#1}
\def \kOanjas #1{鹰的#1}
\def \kOanjalOOkEnji {小鹰的爪}
\def \nyimiileNlOO{小幼蛇}
\def \kandOlEndElOO{小飞机}
\def \leNlOOs #1{小孩子的#1}
\def \leNkundus #1{矮孩子的#1}
\def \nyimiikundus #1{短蛇的#1}
\def \gbomulEndEkundu{矮船}
\def \kais #1{男人的#1}
\def \kOanjaleNfa{幼鹰的父亲}
\def \musugbomu{女人的鱼}
\def \nyimiijaNgbomulEndE{长蛇的船}
\def \musujaNlOOkai{高女人的兄弟}
\def \kaijaNlOOmusu{高男人的姐妹}
\def \kandOjaN {高的天空}
\def \worderNA{形容词接在名词后}
\def \litleunl #1#2#3{除非名词含后缀#2或#3, 否则其词尾为#1}
\def \Ngetmark #1#2{名词(#1)被#2标记}
\def \orAhavan{或形容词[如果存在]}
\def \unlinali{除非其领属关系不可让渡}
\def \bodypart{身体部位}
\def \kinsterm{亲属名称}
\def \thenposs{若为后者, 其置于领属者后}
\def \possessd{隶属者(被所有者)}
\def \possessr{领属者(所有者)}
\def \alienpos #1{可让渡领属关系由领属者和隶属者间的#1表示}
\def \incompoN{在复合名词中}
\def \modihead{左边部分修饰右边部分}
\def \lastonlo{末音节为低调}
\def \tzin #1{#1 (尊称)}
\def \acalf{水房}
\def \acalli{独木舟}
\def \achildef #1{水蓼(#1), 也称辣蓼, 是一种野生植物}
\def \achilli{水蓼}
\def \atl{水}
\def \cacahuatl{可可}
\def \cacahuaatl{可可饮料}
\def \cacahuatetl{可可豆}
\def \callah{村庄}
\def \calli{房子}
\def \chilatl{辣椒水}
\def \chiladef{辣椒水是一种含辣椒的阿兹特克饮品}
\def \chilli{辣椒}
\def \collib{爷爷(祖父)}
\def \collid{祖先}
\def \conehuah{妈妈}
\def \conehuahcapil{妈咪}
\def \conetl{孩子}
\def \oquichconeg{男孩子}
\def \oquichconetl{男孩}
\def \oquichhuah{妻子}
\def \oquichtl{男人/丈夫}
\def \oquichtotolin{公火鸡}
\def \tetl{石头}
\def \tetlah{石子地}
\def \tehuaq{多石地区的居民}
\def \totoltetl{火鸡蛋}
\def \calhuah{房主人}
\def \acalhuah{独木舟夫}
\def \cacahuahuah{可可的拥有者}
\def \tehuag{石头的拥有者}
\def \ahuah{水的拥有者}
\def \conehuaf{孩子的拥有者}
\def \oquichhuaf{丈夫的拥有者}
\def \huah #1{拥有#1的人}
\def \tlah #1{有许多#1的地方}
\def \dimin{指小}
\def \afteroth #1#2#3{在#2后用#1, 否则用#3}
\def \LostSymb{失落的秘符 (The Lost Symbol)}
\def \FIland{芬兰}
\def \FRland{法国}
\def \DEland{德国}
\def \IEland{爱尔兰}
\def \NOland{挪威}
\def \ESland{西班牙}
\def \SEland{瑞典}
\def \books{图书}
\def \ISBNbook{国际标准书号[ISBN]图书}
\def \whence{产地}
\def \cost{价格}
\def \euros #1{#1欧元}
\def \centa #1{#1分}
\def \cents #1{#1分}
\def \checksum{校验码}
\def \lastxsum{最后一位数字永远是校验码}
\def \gifulcod{请写出完整的条码}
\def \fullcode{完整的条码}
\def \prodcode{产品条码}
\def \puzmagaz{益智谜题杂志}
\def \bumpaper{厕纸}
\def \smokelox{烟熏鲑鱼}
\def \farsteak{猪排}
\def \sirsteak{里脊牛排}
\def \chollows{低胆固醇酱}
\def \mopshead{拖把头}
\def \storpack{店内包装}
\def \storefns{店内表记}
\def \inbarcod #1{世界上几乎所有国家都在使用#1条形码“语言”}
\def \anospeak{但没有人说这种“语言”}
\def \subcod{亚级条码}
\def \subcodes #1{它有#1种主要“方言”, 即亚级条码}
\def \nocodnil #1{但本题不涉及亚级条码0, 该亚级条码实际上与更老的#1“语言”完全相同}
\def \isbarcod #1#2{#1是条形码#2}
\def \nobarcod{上图不是条形码}
\def \sibarcod{上图是条形码}
\def \issubcod #1{它属于亚级条码#1}
\def \unsubcod #1{它属于#1的一种可能的亚级条码, 但目前并未使用}
\def \UKbarcod #1{该条码来自一袋英国饼干, 起始数码#1是国家代码, 即英国的系统码}
\def \rebarcod #1#2{通常, 条码的前半部分(#1)标记生产者, 而后半部分(#2) 由生产者选择以标记产品}
\def \gridrite{为便于观察, 本题放大了条码的机读部分并将其转写在细格内, 列于原条码右侧}
\def \morenums{下列是部分系统码}
\def \barcodAf #1{下列是关于#1的部分信息 (乱序排列)}
\def \barcodAz{请将条码与信息对应, 并填补空缺的信息}
\def \barcodBa #1{在答题纸提供的细格中绘制 (虚构的) 条码 #1}
\def \barcodBe{为了帮助解题, 部分细格已填充}
\def \dagblaNO #1{下图所示的条码来自挪威报纸#1}
\def \hvaderNO{挪威的系统码 (国家代码) 为何}
\def \barframe #1#2{单位宽度为 #1 (两端) 和 #2 (中间)的条纹式样构成了2段6位数码}
\def \digitcod #1#2{每位数码表示为宽度为#1的4条条纹, 其总宽度为#2}
\def \threecod #1#2{每位数码对应着3个代码, 其中1个(#1)用在右边, 2个(#2)用在左边}
\def \subcodep #1#2{左侧的#1和#2式样给出亚级条码}
\def \allstart #1{每种式样由#1起始}
\def \conthree #1{恰包含3个#1}
\def \rightway{这表示条码正向朝上}
\def \elsemirr #1#2{否则条码会由#1(即#2的镜像)起始}
\def \feallbar #1{本题涉及了除#1之外的所有式样}
\def \onpriceA{仅有重量不定的产品(例如肉类和芝士等), 其条码包含价格信息}
\def \onpriceB{其它产品的价格信息由店内计算机系统检索获得}
\def \onpriceC #1{这些产品由店内生产 (#1), 因而不具有标准格式}
\def \onpriceD{但对于本题中涉及的2种店内产品, 校验码前的最后4位数码表示价格}
\def \upsidown{该条码为反向}
\def \startwiB #1#2{它由#1[而非#2]起始}
\def \mustturn{因而必须将其上下颠倒后从后向前填写}
\def \teamiont{团体赛说明}
\def \teamiona #1{团体赛共3小时, 题目装于写有队名和#1的信封内}
\def \teamionb #1{比赛开始后30分钟须提高首份解答(#1)}
\def \ansinsix{对问题的答案或思路}
\def \nodyssey{请勿包含解题的尝试性过程}
\def \iconvert{问题放于信封中}
\def \teamionc #1{其后, 考场人员会送来一个标有 #1的提示信封, 其中包含部分问题的答案}
\def \teamionz{解题思路和贴士}
\def \Insights{解题思路}
\def \teamiond #1#2{同样的要求适用于#1, 比赛开始后#2分钟}
\def \barfinal #1{然而比赛开始后 #1分钟将不再给予提示, 而将给出完整解答}
\def \teamione{在越早的半小时间隔提交正确答案和思路, 得分越高}
\def \teamionf #1{每过半小时, 得分将减少约#1分}
\def \teamiong{同样的内容毋需多次提交}
\def \latignor{之后提交的相同内容将被忽视}
\def \indepont{每次提交将独立评分}
\def \teamionh{正确的答案将获得加分, 而错误的答案将导致扣分}
\def \teampteg{例如, 若第30分钟时提交的错误答案而在第90分钟修正}
\def \teamioni{在第30分钟的评判中会扣分, 而在第90分钟的评判中会加分}
\def \teamionj{首张答题纸会模糊地描述将被当做提示给出的思路}
\def \teamionk{除这些外, 其它的思路也有可能得分}
\def \teamionl{之前提交的内容将不允许查看}
\def \makenots{因此, 务必对打算提交的内容做好笔记, 以备后需}
\def \teamionm{对问题的完整描述和所有答案将被印在一张纸上}
\def \teamionn{因而填写答案不会产生额外负担}
\def \disminun{在起始时刻分发}
\def \disminut #1{在第#1分钟分发}
\def \colminut #1{在第#1分钟收集}
\def \ansminut #1{第#1分钟的答案}
\def \justhint{这只是一个提示, 而非可得分的思路}
\def \follhint{以下的思路将在之后以提示的形式给出}
\def \signifof #1{#1的意义}
\def \soonhint #1{诗歌格式的限制, 以及(将作为3条提示给出)#1的运用}
\def \introSkr{下列是10行有误的梵语诗歌}
\def \werewell{这些诗句原本完全正确}
\def \mumacrod{但本题删去了其中5个长音符}
\def \vomacrob{另外增加了4个长音符}
\def \deschang{更改了3个字母, 并删去了2个单词}
\def \onewhole{最终只有1行毫无改动}
\def \sylstand{诗行中没有增加或删减音节}
\def \sinodeld{当然, 被删去的单词除外}
\def \Addmacra{增加长音符的单词}
\def \Delmacra{删去长音符的单词}
\def \Delwords{删去的单词}
\def \dLetters{改动的字母}
\def \redelmot{请恢复2个删去的单词}
\def \remodlet{还原3个改动的字母}
\def \redothem{移除4个额外的长音符, 并还原5个删去的长音符}
\def \canrecon #1#2{#2中#1上的长音符可以基于诗歌格律的原因恢复}
\def \nonrecon #1#2{但为了将#2中#1上额外的长音符除去, 必须认识这个词}
\def \orcompar #1{或者将其与#1比较}
\def \fortpoet{所幸, 本题诗歌中除上述的长音符外, 其它改动均可在毫无梵语储备知识的情况下修复}
\def \trawrong{题中的翻译对应替换了字母, 删除了单词(但尚未改动任何长音符)的梵语诗歌}
\def \yanother{题中还涉及一条与格律有关的字母转写规则, 有待解题者发现}
\def \whataffd{题干中未阐述的转写规则为何}
\def \thingist #1{额外的转写规则: #1是长元音}
\def \eolength #1{#1是长元音}
\def \barnomac{尽管转写时未使用长音符}
\def \offmacra #1{以下是一条省略了长音符的梵语记忆口诀: #1}
\def \mnemodef{记忆口诀是指有助于记住某事物的单词或语句}
\def \pimnemon{山巅一寺一壶酒}
\def \whatsyll #1{哪些音节是#1}
\def \whatsyli #1#2{#2中的哪些音节是#1}
\def \isquaple #1#2{#1含四#2}
\def \quadusef #1{第#1行中“包含4个某物”是有用的解题信息 (即便不知“某物”为何)}
\def \whethere #1#2{若#1存焉, #2亦附也}
\def \ifatends #1#2#3#4{#2个#3尾部之#1名曰#4也}
\def \vadethet #1{名曰#1也}
\def \twolines #1#2{两行#1称曰#2也}
\def \ifthrees #1#2#3#4{若有缘逢一对#1, 一#2, 与一#3, 此乃#4也}
\def \thinerst #1#2#3{#1亦属之#2, 然其首为#3也}
\def \bhujpray{蛇之舞动}
\def \gajagati{象之步履}
\def \pramANik{小节拍}
\def \pramANix{小节拍}
\def \Indravaz{帝释霹雳}
\def \Indravaj{帝释霹雳}
\def \Upendvaj{侏儒霹雳}
\def \Upendrau{帝释(又译因陀罗) 及其弟侏儒(又译优宾陀罗)是印度教神明}
\def \CaCikalA{月升之期}
\def \vidymAlA{闪电之环}
\def \madhumat{蜜糖}
\def \pancAmar{五牦尾之扇}
\def \hintcomp #1#2{建议: 请比较第#1行, 以及第#2行}
\def \hintread{建议: 请所有参赛者仔细阅读题目的每一部分和每一条提示, 其中部分内容可能与答案存在意想不到的关联}
\def \kendinna{若抄写员懂得梵语, 但不清楚这条记忆口诀}
\def \resthint #1#2{其更可能将(例如)#1抄写成#2而非其它基础梵语单词}
\def \poetinte{从诗歌的角度看, 部分格律的名称与其格律间存在有趣的对应关系}
\def \eginluna #1{例如, 月逐渐变圆时, 从新月到满月的周期为#1天}
\def \nonguess{除非参赛者能仅仅从诗歌格律的名称猜出其内容}
\def \writback{若有需要, 请使用答题纸反面}
\def \selfdesc{每行诗描述其使用的格律}
\def \divindiv{划分音节时, 不考虑单词划分}
\def \sylladiv #1#2#3#4{#1类型的序列划分为#2; #3类型的序列划分为#4}
\def \defSguru #1{某音节为#1, 当且仅当其包含长元音或双元音, 或者以辅音结尾}
\def \descmetr #1{每种格律可被描述为#1中一段 \emph{独特的}辅音序列}
\def \adelword{其中一个被删除的单词是}
\def \thatlett #1{#1, }
\def \addedlen{添加了1个长音符}
\def \deledlen{的长音符被删去}
\def \inWord #1{单词#1, }
\def \Lineword{诗行}
\def \inLine #1{第#1行}
\def \Twosylls{音节}
\def \Syllsare #1#2#3{第#1与#2音节是#3}
\def \inSyll #1{第#1音节, }
\def \teislett #1{第2个#1, }
\def \lastlett #1{最后的#1, }
\def \isbroken{不正确}
\def \wasfirst #1{原本为#1}
\def \sentowas #1#2{例如, 语句#1原本为#2}
\def \mnemexpl #1#2{#1的其它音节分别表示\emph{独特的} \emph{3个}或#2音节序列}
\def \mnemexpn #1#2#3{#2的前#1个音节分别表示该音节和其后2个音节中#3音节的式样}
\def \macresto{恢复其长音符后}
\def \makemnem{格律的记忆口诀}
\def \tritridu #1{由音节三三成组构成. 末尾(至多)2个额外音节用#1标记}
\def \standfor #1{表示#1}
\def \infacteo #1#2{事实上, 这两个音素原本是#1, 而现有的双元音原本是#2}
\def \ABname{Aleksandrs Berdičevskis}
\def \AHname{Adam Hesterberg}
\def \APname{Alexander Piperski}
\def \BBname{Bozhidar Bozhanov}
\def \BIname{Boris Iomdin}
\def \DGname{Dmitry Gerasimov}
\def \HDname{Hugh Dobbs}
\def \IDname{戴谊凡}
\def \LFname{Liudmila Fedorova}
\def \MRname{Maria Rubinstein}
\def \PSname{Pavel Sofroniev}
\def \SBname{Svetlana Burlak}
\def \SGname{Stanislav Gurevich}
\def \TTname{Todor Tchervenkov}
\def \XGname{Ksenia Gilyarova}
\def \OKname{Olga Kuznetsova}
\def \CQTname{曹起曈}
\def \LMSname{刘闽晟}
\def \edinames{\ABname, \BBname, \SBname, \IDname, \HDname, \LFname, \DGname, \XGname, \SGname, \AHname\ (\edinchef), \BIname, \APname, \MRname, \PSname, \TTname}
\def \enqueten{姓名}
\def \enquetep{座位号}
\def \leafword #1{页 \##1}
\def \enquetea{你做了哪些题目?}
\def \enqueteb{你最喜欢哪道题?}
\def \enquetec{你觉得哪道题最难?}
\def \enqueted{你觉得哪道题最简单?}
\def \sparecop{若需要这张答题纸的副本, 请示意考场人员}
\def \quoted #1{“#1”}
\def \zagrad #1{(#1.)}
\def \decpoint{.}
\def \et{和}
\def \ab{或}
\def \au{或}
\def \daaracht{则}
\def \is{为}
\def \an #1{#1}
\def \An #1{#1}
\def \whowroti{\CQTname, \LMSname}
\def \whowrotj{\AHname}
